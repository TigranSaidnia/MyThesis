\section{IR and Collinar Divergences}
Beyond the LO (Leading order) diagrams it happens singularities. To discuss about these consider first the process $ e^- e^+ \rightarrow q\bar{q}g $

\begin{figure}[ht!]
\centering
\includegraphics[width=0.85\textwidth]{images/Intro/IRCol.png}
\end{figure}

In order to calculate the cross section of this diagram, we have to consider the gluon emission from the antiquark. Since the calculation is quite long, we concentrate on the final result:
\begin{figure}[ht!]
\centering
\includegraphics[width=0.85\textwidth]{images/Intro/IRColMatrix.png}
\caption{Left diagram $  e^- e^+ \rightarrow qg\bar{q} $ and right $ e^- e^+ \rightarrow q\bar{q}g $}
\end{figure}

\begin{equation}
\begin{split}
&A= \frac{\bar{u}(k_1)(-ig_s\gamma^{\nu}\times T^a)[-i(\not{k_1}+\not{k_3})](-iee_q \gamma^{\mu})v(k_2){\epsilon_{\mu}}^{\lambda_1}{\epsilon_{\nu}}^{\lambda_2*}}{(k_1 + k_3)^2}\\ 
&- \frac{\bar{u}(k_1)(-iee_q \gamma^{\mu})[i(\not{k_2}+\not{k_3})](-ig_s\gamma^{\nu}\times T^a)v(k_2){\epsilon_{\mu}}^{\lambda_1}{\epsilon_{\nu}}^{\lambda_2*}}{(k_1 + k_3)^2}\\
\Rightarrow &A=-g_s T^a[ \frac{\bar{u}\:\not{\epsilon}\:(\not{k_1}+\not{k_3})\:\Gamma \:v}{(k_1 + k_3)^2} - \frac{\bar{u}\:\Gamma\:(\not{k_2}+\not{k_3})\:\not{\epsilon} \:v}{(k_2 + k_3)^2}] \:\:\:\:\text{with} \:\:\Gamma=(-iee_q \gamma^{\mu}){\epsilon_{\mu}}^{\lambda_1}
\end{split}
\end{equation}
Under consideration that the partons are on-shell, we get:
\begin{equation}
 A=-g_s T^a[ \frac{\bar{u}\:\not{\epsilon}\:(\not{k_1}+\not{k_3})\:\Gamma \:v}{2k_1 \cdot k_3} - \frac{\bar{u}\:\Gamma\:(\not{k_2}+\not{k_3})\:\not{\epsilon} \:v}{2k_2 \cdot k_3}]
\end{equation}
In the soft limit with $k_0 \rightarrow 0$ we can factorize $ A_{soft} $ the amplitude in two parts:
\begin{equation}
 A=-g_s T^a[ \frac{k_1\:{\epsilon}}{k_1 \cdot k_3} - \frac{k_2\:{\epsilon}}{k_2 \cdot k_3}] A_{born} \:\:\:\:\:\:\:\:\:\:\:\:\:\:\:\:\:\text{with}\:\: A_{born}= \bar{u}\: \Gamma \:v
\end{equation}
 Which one contains all information about colour and momenta and $ A_{born} $ with all spin information.
If one calculates the cross section for it, one gets:
\begin{equation}
\begin{split}
A=&C_F g_s^2 \sigma^{born} \int \frac{d^3 k}{2k_0 (2{\pi})^3} 2(\frac{k_1 \cdot k_2}{(k_1 \cdot k_3)(k_2 \cdot k_3)})\\ 
&C_F g_s^2 \sigma^{born} \int dcos\: \theta\: \frac{d k_0}{k_0} \frac{4}{(1-cos\: \theta)(1+cos\: \theta)}
\end{split}
\end{equation}
We define the energy fraction by:
\begin{equation}
x_i = \frac{2E_i}{\sqrt{s}}=\frac{2q\: \cdot\: k_i}{s}
\end{equation}
One can show that $ \sum x_i =2  $ and thus, that only two of the $ xi $ are independent.

The final result is:
\begin{equation}
\frac{d^2 \sigma}{dx_1 dx_2}= (\frac{4\pi \alpha}{s})\sum {e_i}^2 
\frac{2\alpha_s}{3\pi} \frac{{x_1}^2+{x_2}^2}{(1-x_1)(1-x_2)}
\end{equation}

There are three singularities in regard with the final result. 
If the emitted photon is collinear to the outgoing quark or anti-quark $ (x_1 \rightarrow 1 \:\text{or}\: x_2 \rightarrow 1) $ and When the emitted gluon is very soft $ (x_1 \rightarrow 1\: \text{and}\: x_2 \rightarrow 1 )$.
The singularities come from the quark propagator in each diagram. The denominators contain according Feynmann rules terms with $\sim \frac{1}{(k_i + k_j)^2}  $. We can eliminate the quark mass under on-shell condition so that:
\begin{equation}
\frac{1}{(k_i + k_j)^2}=\frac{1}{2k_i \cdot k_j}=\frac{1}{2E_iE_j(1-cos\theta_{ij})}=\frac{1}{s(1- x_k)}
\end{equation} 
One can show all possibilities for three partons through a triangle:
\pagebreak

\begin{figure}[ht!]
\centering
\includegraphics[width=0.85\textwidth]{images/Intro/triangle.png}
\end{figure}

fortunately, According to KLN-Theorem one can eliminate the real singularities by adding the virtual contributions to them. The sum over real and virtual contributions in the phase space integral must be finite. 
\begin{figure}[ht!]
\centering
\includegraphics[width=0.85\textwidth]{images/Intro/virtual.png}
\end{figure}




We will use deep inelastic scattering (DIS) to show how the infrared singularities are absorbed in the parton distributions.

\newpage
\section{Factorisation}
The hadron hadron scattering can be written as:
\begin{equation}
\sigma = \sum_{ij} \int dx_1 dx_2 f_i(x_1, \mu^2)f_j(x_2, \mu^2) \sigma_{ij}(x_1, x_2, Q^2/\mu^2... )
\end{equation}
\begin{figure}[h!]
\centering
\includegraphics[width=0.85\textwidth]{images/Intro/Hard.png}
\end{figure}
Here the (arbitrary) factorisation scale $ \mu $ can be thought of
as the scale which separates the long and short-distance physics.
Roughly speaking, a parton with a transverse momentum less
than $ \mu $ is then considered to be part of the hadron structure and
is absorbed in the parton distribution. Partons with larger transverse
momenta participate in the hard scattering process with a
short-distance partonic cross-section.
The factorisation theorem also applies to deep inelastic scattering. The DIS cross section can be written as:
\begin{equation}
\frac{d^2 \sigma}{dx dQ^2}=\frac{4\pi \alpha^2}{x Q^4}[(1-y)F_2 (x, Q^2)+xy^2 F_1(x, Q^2)]
\end{equation}
In this case we need to introduce the structure function, is defined as the charge weighted sum of the parton momentum densities, the probability that the parton carries a momentum
fraction x. The index i denotes the quark. 
flavour.
\begin{equation}
{{F}_2}^{exp} (x)= \sum_i {e_i}^2 x f_i(x)
\end{equation}
The evolution of a quark
distribution due to gluon radiation and is called the DGLAP
evolution equation.
\begin{equation}
\frac{\partial f(x, \mu^2) }{\partial \: ln \:\mu^2}=
\frac{\alpha_s}{2\pi}\int_{x}^{1}\frac{dy}{y} f(y, \mu^2) P_{qq}(\frac{x}{y}+O({\alpha_s}^2)
\end{equation}
There three more spiriting possibilities according to below diagrams.

\begin{figure}[h!]
\centering
\includegraphics[width=0.85\textwidth]{images/Intro/spiliting.png}
\end{figure}

\begin{equation}
	\left.\begin{aligned}
\langle\:\hat{P_{qq}}\rangle &= C_F[\frac{1+z^2}{1-z}-\varepsilon(1-z)]\\
\langle\:\hat{P_{gq}}\rangle &= T_R[1-\frac{2z(1-z)}{1-\varepsilon}]\\
\langle\:\hat{P_{qg}}\rangle &= C_F[\frac{1+(1-z)^2}{z}-\varepsilon z]\\
\langle\:\hat{P_{gg}}\rangle &= 2C_A[\frac{z}{1-z}+\frac{1-z}{z}+z(1-z)]
\end{aligned}
	\right\}
	\quad \text{splitting functions}
\end{equation}
\newpage
\section{splitting/subtraction method}

\section{Catani-Seymour Formalism}