\begin{figure}[ht!]
\centering
\includegraphics[width=0.85\textwidth]{images/QQ/qqg-diagrams.png}
\end{figure}
First we are going to consider a daughter quark from the splitting of a parent quark into a
quark and a gluon with an arbitrary spectator like an anti-quark, see the left picture above. where $ q_j+q $ is the momentum of the quark before splitting, $q$ the momentum of the gluon and $q_j$ of daughter quark respectively. The momentum of the spectator is $ q_j $. The distinction between daughter and parent vanishes, when the gluon becomes soft,  and a
singularity develops. The other possibility to get a singularity is surely if the gluon will be collinear to quark. The splitting functions are flavour independent since the strong interaction is flavour independent. Furthermore,
leading order splitting cannot change the flavour of a quark, thus we can write for the splitting functions In leading order QCD:
\begin{equation}
P_{{\bar{q_i}}{\bar{q_j}}}=P_{{q_i}{q_j}}=P_{{q}{q}} \delta_{ij}
\end{equation}
For this aim we have to take any diagram to the account which can have the same splitting. Since there is no distinction between quark and anti-quark, one can imagine exact the same splitting variation for anti-quark with a quark as a spectator, see the right picture above.


\pagebreak
\section{Matrix element of a quark with a gluon radiation $ |M_1|^2 $}

\begin{figure}[h!]
\centering
\includegraphics[scale=0.7]{images/QQ/qgqbarM.png}
\end{figure}
If one simply calculates the amplitude of this diagram, one gets:
\begin{equation}
M_1 = [{\bar{u}}_{\sigma}(q_i) (-ig_s \gamma^{\mu}\times {[T^a]_o}^l)  \frac{i(\not{q_i} + \not{q})}{(q_i + q)^2} {\varepsilon^{\lambda_1}}_{\mu} (q)]\: [{v}_{\tau}(q_j)]
\end{equation}
For the quadratic matrix element we need the dagger of $ M_1 $ as well.
\begin{figure}[h!]
\centering
\includegraphics[scale=0.7]{images/QQ/qgqbarMDega.png}
\end{figure}

\begin{equation}
{M_1}^{\dagger} = [\frac{-i(\not{q_i} + \not{q})}{(q_i + q)^2} \:  (ig_s \gamma^{{\mu}^{\prime}}\times {[T^b]_{o\:^{\prime}}}^k) \: u_{{\sigma}^{\prime}}(q_i) \: {\varepsilon^{\lambda_2}}_{{\mu}^{\prime}} (q)][{\bar{v}}_{{\tau}^{\prime}}(q_j)]
\end{equation}
\pagebreak
\begin{figure}[h!]
\centering
\includegraphics[width=0.85\textwidth]{images/QQ/qgqbarMSquer.png}
\end{figure}
After multiplying $ {M_1}^{\dagger} $ and $ {M_1} $ we get the desired result.
\begin{equation}
\begin{split}
|M_1|^2=M_1\:{\color[RGB]{255,0,0} {M_1}^{\dagger}} = [{\bar{u}}_{\sigma}(q_i)\: (-ig_s \gamma^{\mu}\times {[T^a]_o}^l) \: \frac{i(\not{q_i} + \not{q})}{(q_i + q)^2}\:\: {\varepsilon^{\lambda_1}}_{\mu} (q)] [{v}_{\tau}(q_j)]\: \\
\quad\quad\quad\quad\quad\quad\quad\quad\:\:{\color[RGB]{255,0,0}[\frac{-i(\not{q_i} + \not{q})}{(q_i + q)^2} \:  (ig_s \gamma^{{\mu}^{\prime}}\times {[T^b]_{o\:^{\prime}}}^k) \: u_{{\sigma}^{\prime}}(q_i) \: {{\varepsilon^{\lambda_2}}_{{\mu}^{\prime}}}^* (q)][{\bar{v}}_{{\tau}^{\prime}}(q_j)]}
\end{split}
\end{equation}
Now it's time to connect those terms which are related to each other.

\begin{equation}
\begin{split}
|M_1|^2=[\frac{-i(\not{q_i} + \not{q})}{(q_i + q)^2} \:
 \:  (ig_s \gamma^{{\mu}^{\prime}}\times {[T^b]_{o\:^{\prime}}}^k) \: {\bar{u}}_{\sigma}(q_i)\:u_{{\sigma}^{\prime}}(q_i) \: {{\varepsilon^{\lambda_2}}_{{\mu}^{\prime}}^* (q) {\varepsilon^{\lambda_1}}_{\mu} (q)} \\
\times (-ig_s \gamma^{\mu}\times {[T^a]_o}^l) \: \frac{i(\not{q_i} + \not{q})}{(q_i + q)^2} ]
[{\bar{v}}_{{\tau}^{\prime}}(q_j) {v}_{\tau}(q_j)]
\end{split}
\end{equation}

Sum over the lorenz index $({\sigma},{\sigma}^{\prime})$ and $({\tau},{\tau}^{\prime})$ and spin addition relation leads to:
 
\begin{equation}
\begin{split}
\displaystyle\sum\limits_{{\sigma},{\sigma}^{\prime}} {\bar{u}}_{\sigma}(q_i)\:u_{{\sigma}^{\prime}}(q_i) = \not{q_i} \delta^{{o}{o}^{\prime}},\\
\displaystyle\sum\limits_{{\tau},{\tau}^{\prime}} {\bar{v}}_{\tau}(q_j)\:v_{{\tau}^{\prime}}(q_j) = \not{q_j} \delta^{{f}{f}^{\prime}}
\end{split}
\end{equation}
Sum over polarization index $({\lambda_{1}},{\lambda}_{2})$ :
\begin{equation}
\begin{split}
 \displaystyle\sum\limits_{{\mu},{\mu}^{\prime}} {{\varepsilon^{\lambda_2}}_{{\mu}^{\prime}}^* (q) {\varepsilon^{\lambda_1}}_{\mu} (q)} = -g_{{\mu}{\mu}^{\prime}} \delta^{{a}{b}}
\end{split}
\end{equation}
The matrix element will be simplified with: 
\begin{equation}
\begin{split}
|M_1|^2=\frac{-g_s^2  {[T^a]_{o}}^k \: {[T^a]_o}^l }{(q_i + q)^2 (q_i + q)^2}
[(\not{q_i} + \not{q}) \:
 \:  \gamma^{{\mu}^{\prime}} \: \not{q_i} \: g_{{{\mu}^{\prime}}{\mu}} 
\gamma^{\mu} \: (\not{q_i} + q)]
[\not{q_j}]
\end{split}
\end{equation}

When we contract all related indices we will be actually able to make a statements about the last result.
\begin{equation}
\begin{split}
|M_1|^2=\frac{-g_s^2  {[T^a]_{o}}^k \: {[T^a]_o}^l }{(q_i + q)^2 (q_i + q)^2}
[(\not{q_i} + \not{q}) \:
 \:  \gamma^{{\mu}^{\prime}} \: \not{q_i} \: 
\gamma_{{\mu}^{\prime}} \: (\not{q_i} + q)]
[\not{q_j}]
\end{split}
\end{equation}

In other words we expect the tree level diagram from LO and a number:\\
\begin{figure}[h!]
\centering
\includegraphics[width=0.85\textwidth]{images/QQ/expectationqg-qbar.png}
\end{figure}
Which graphically means:\\
\begin{equation}
\begin{split}
|M_1|^2=\frac{-g_s^2  {[T^a]_{o}}^k \: {[T^a]_o}^l }{(q_i + q)^2 (q_i + q)^2}
[\not{P_i}]
[\not{P_j}]\:\otimes \: {\color[RGB]{255,0,0} (a\: complex\: number)}
\end{split}
\end{equation}  
Let's calculate the contribution and compare the final result with this expectation:
\begin{equation}
\begin{split}
N=&: \gamma^{{\mu}^{\prime}} \not{q_i} \: \gamma_{{\mu}^{\prime}} = {q_{i\sigma}} \: \gamma^{{\mu}^{\prime}} \gamma^{\sigma} \:\: \gamma_{{\mu}^{\prime}}\\
=& \: {q_{i\sigma}} \: (\lbrace{\gamma^{{\mu}^{\prime}}}, {\gamma^{\sigma}}\rbrace \: - {\gamma^{\sigma}}{\gamma^{{\mu}^{\prime}}})\gamma_{{\mu}^{\prime}}\\
=& \:{q_{i\sigma}} \: 2g^{{{\mu}^{\prime}}{\sigma}} \: \gamma_{{\mu}^{\prime}} \: - \:d\:{\gamma^{\sigma}}\\
=& \:(2-d) \not{q_i}
\end{split}
\end{equation}
Simplification of the bracket:
\begin{equation}
\begin{split}
|M_1|^2=-(2-d)\:\frac{g_s^2  {[T^a]_{o}}^k \: {[T^a]_o}^l }{(q_i + q)^2 (q_i + q)^2}
[(\not{q_i} + \not{q}) \:
 \:\not{q_i} \: 
 \: (\not{q_i} + q)]
[\not{q_j}]
\end{split}
\end{equation}

\begin{equation}
\begin{split}
|M_1|^2=-(2-d)\:\frac{g_s^2  {[T^a]_{o}}^k \: {[T^a]_o}^l }{(q_i + q)^2 (q_i + q)^2}
[\not{q_i} \not{q_i} \not{q_i} \: + \: \not{q_i} \not{q_i} \not{q} \: + \: \not{q} \not{q_i} \not{q_i} \:+\: \not{q} \not{q_i} \not{q}]
[\not{q_j}]
\end{split}
\end{equation}

Momenta are on-shell, so:
\begin{equation}
\begin{split}
\not{q_i}\: \not{q_i} &= {q_i}^2= {m_i}^2\\
\not{q} \: \not{q} &= {q}^2= {m}^2\\
\not{q_j}\not{q_j} &= {q_j}^2= {m_j}^2
\end{split}
\end{equation}

we can first neglect the mass of patrons:

\begin{equation}
\begin{split}
|M_1|^2=-(2-d)\:\frac{g_s^2  {[T^a]_{o}}^k \: {[T^a]_o}^l }{(2q_i q)(2q_i q)}
[\not{q} \not{q_i} \not{q}]
[\not{q_j}]
\end{split}
\end{equation}
Here we need to make the terms in the brackets simpler and:
\begin{equation}
\begin{split}
L=& \not{q} \not{q_i} \not{q} =\not{q}[{q_{i\sigma}} q_{\mu} \: (\lbrace{\gamma^{\mu}}, {\gamma^{\sigma}}\rbrace - {\gamma^{\sigma}}{\gamma^{\mu}})]\\ 
=& \not{q}[2{q_{i}}^{\mu} q_{\mu} - {q_{i\sigma}}q_{\mu}{\gamma^{\mu}}{\gamma^{\sigma}}\\
=& \not{q} (2q_i q)-q_{\mu}{q_{i\sigma}}q_{\mu}[{\gamma^{\mu}}{\gamma^{\mu}}{\gamma^{\sigma}}]\\
=& \not{q} (2q_i q)-q_{\mu}{q_{i\sigma}}q_{\mu}[\frac{{\gamma^{\mu}}{\gamma^{\mu}}}{2} +\frac{{\gamma^{\mu}}{\gamma^{\mu}}}{2}]{\gamma^{\sigma}}\\
=& \not{q} (2q_i q)-q_{\mu}{q_{i\sigma}}q_{\mu}[g^{{\mu}{\mu}}]{\gamma^{\sigma}}\\
=& \not{q} (2q_i q)-q_{\mu}{q_{i\sigma}}q^{\mu}{\gamma^{\sigma}}
=\not{q} (2q_i q)-q^2 \not{q_i}\\
=& \not{q} (2q_i q)
\end{split}
\end{equation}
After inserting the last result of $ L $ and simplify the term $ (2q_i q) $ from the denominator and nominator, we get:
\begin{equation}
\begin{split}
|M_1|^2=-(2-d)\:\frac{g_s^2  {[T^a]_{o}}^k \: {[T^a]_o}^l }{2y(1-2z+2z^2)(p_i \cdot p_j)}
[\not{q}]
[\not{q_j}]
\end{split}
\end{equation}
Now we are going to use the parametrisation from equation (1) to reduce the 3-member matrix element to 2-member and take out the singularity term from the amplitude.
\begin{equation}
\begin{split}
|M_1|^2=(d-2)\:\frac{g_s^2  {[T^a]_{o}}^k \: {[T^a]_o}^l }{2y(1-2z+2z^2)(p_i \cdot p_j)}
[(1-z) \not{p_i}+zy \not{p_j} - \sqrt{zy(1-z)} \not{{m}_{\bot}}]
[(1-y) \not{p_j}]
\end{split}
\end{equation}
Multiplying the both sides 
\begin{equation}
\begin{split}
|M_1|^2=(d-2)\:\frac{g_s^2  {[T^a]_{o}}^k \: {[T^a]_o}^l }{2y(1-2z+2z^2)(p_i \cdot p_j)}
[(1-z)(1-y) \not{p_i}\not{p_j} \\
+zy(1-y) \not{p_j}\not{p_j} + (1-y)\sqrt{zy(1-z)} \not{{m}_{\bot}}\not{p_j}]
\end{split}
\end{equation}
Under consideration of the fact that $ p_i $ and $ p_j $ are the on-shell momenta of the emitter and spectator partons, we can ignore the terms with $ \not{p_i} \not{p_i} $ and $ \not{p_j} \not{p_j} $.
The $ {p_i} \cdot  {m}_{\bot} $ and $ {p_j} \cdot  {m}_{\bot} $ are always $ 0 $ because the $ p_i $ and $ p_j $ are lightlike, i.e. zero transverse component. So those terms can be neglected.


\begin{equation}
\begin{split}
|M_1|^2=\frac{g_s^2  {[T^a]_{o}}^k \: {[T^a]_o}^l }{(p_i \cdot p_j)}
[\not{p_i}][\not{p_j}]\otimes\frac{(d-2)(1-z)(1-y)}{2y(1-2z+2z^2)}
\end{split}
\end{equation}

As discussed, we get a contribution from the LO a complex number. As you can see, the number is just for $ y \rightarrow 0 $ singular and not for $ z \rightarrow 1 $.

\newpage

\section{Matrix element of an anti-quark with a gluon radiation $ |M_2|^2 $}

%\begin{figure}[h!]
%\centering
%\includegraphics[scale=0.7]{images/QQ/qbargqM.png}
%\end{figure}
%
%\begin{equation}
%M_2 = [\frac{i(\not{q_j} + \not{q})}{(q_j + q)^2} (-ig_s \gamma^{\nu}\times {[T^c]_f}^m) \:{v}_{\tau}(q_j)\: {\varepsilon^{\lambda_3}}_{\nu} (q)]\: [{u}_{\sigma}(q_i)]
%\end{equation}
%\begin{figure}[h!]
%\centering
%\includegraphics[scale=0.7]{images/QQ/qbargqMDega.png}
%\end{figure}
%\begin{equation}
%M_2^{\dagger} = [\bar{v}_{{\tau}^{\prime}}(q_j) \: (ig_s \gamma^{{\nu}^{\prime}}\times {[T^d]_{f^{\prime}}}^n) \: \frac{-i(\not{q_j} + \not{q})}{(q_j + q)^2} \: {\varepsilon^{\lambda_4}}_{{\nu}^{\prime}} (q)]\: [\bar{u}_{{\sigma}^{\prime}}(q_i)]
%\end{equation}
the same procedure is used to obtain the matrix element for an anti-quark with a single gluon emission.
\begin{figure}[h!]
\centering
\includegraphics[width=0.85\textwidth]{images/QQ/qbargqMSquer.png}
\end{figure}

\begin{equation}
\begin{split}
|M_2|^2=M_2\:{\color[RGB]{255,0,0} {M_2}^{\dagger}} = [\frac{i(\not{q_j} + \not{q})}{(q_j + q)^2} (-ig_s \gamma^{\nu}\times {[T^c]_f}^m) \:{v}_{\tau}(q_j)\: {\varepsilon^{\lambda_3}}_{\nu} (q)]\: [{u}_{\sigma}(q_i)]\: \\
\quad\quad\quad\quad\quad\quad\quad\quad\:\:{\color[RGB]{255,0,0}[\bar{v}_{{\tau}^{\prime}}(q_j) \: (ig_s \gamma^{{\nu}^{\prime}}\times {[T^d]_{f^{\prime}}}^n) \: \frac{-i(\not{q_j} + \not{q})}{(q_j + q)^2} \: {\varepsilon^{\lambda_4}}_{{\nu}^{\prime}} (q)]\: [\bar{u}_{{\sigma}^{\prime}}(q_i)]}
\end{split}
\end{equation}


\begin{equation}
\begin{split}
|M_2|^2 =\frac{g_s^2 \: {[T^c]_f}^m \: {[T^d]_{f^{\prime}}}^n }{(q_j + q)^2 (q_j + q)^2} [(\not{q_j} + \not{q}) \gamma^{\nu}  \:{v}_{\tau}(q_j)\bar{v}_{{\tau}^{\prime}}(q_j)\: {\varepsilon^{\lambda_3}}_{\nu} (q){\varepsilon^{\lambda_4}}_{{\nu}^{\prime}}  (q) \gamma^{{\nu}^{\prime}}(\not{q_j} + \not{q})]\: \\
[{u}_{\sigma}(q_i) ]
\: [\bar{u}_{{\sigma}^{\prime}}(q_i)]
\end{split}
\end{equation}

and after sum over the lorenz and polarization indexes like $({\sigma},{\sigma}^{\prime})$, $({\tau},{\tau}^{\prime})$ and $({\lambda_{3}},{\lambda}_{4})$ as well and using the spin addition relation:
 
%\begin{equation}
%\begin{split}
%\displaystyle\sum\limits_{{\sigma},{\sigma}^{\prime}} {\bar{u}}_{\sigma}(q_i)\:u_{{\sigma}^{\prime}}(q_i) = \not{q_i} \delta^{{o}{o}^{\prime}},\\
%\displaystyle\sum\limits_{{\tau},{\tau}^{\prime}} {\bar{v}}_{\tau}(q_j)\:v_{{\tau}^{\prime}}(q_j) = \not{q_j} \delta^{{f}{f}^{\prime}}
%\end{split}
%\end{equation}
%
%\begin{equation}
%\begin{split}
% \displaystyle\sum\limits_{{\nu},{\nu}^{\prime}} {{\varepsilon^{\lambda_4}}_{{\nu}^{\prime}}^* (q) {\varepsilon^{\lambda_3}}_{\nu} (q)} = -g_{{\nu}{\nu}^{\prime}} \delta^{{c}{d}}
%\end{split}
%\end{equation}

\begin{equation}
\begin{split}
|M_2|^2 =\frac{g_s^2 \: {[T^c]_f}^m \: {[T^c]_{f}}^n }{(q_j + q)^2 (q_j + q)^2} [(\not{q_j} + \not{q}) \gamma^{\nu}  \:\not{q_j}\: (-g_{{\nu}{{\nu}^{\prime}}}) \gamma^{{\nu}^{\prime}}(\not{q_j} + \not{q})]\: 
[\not{q_i} ]
\end{split}
\end{equation}

Analogous to the last calculation from the previous section:

\begin{equation}
\begin{split}
|M_2|^2 =(d-2) \frac{g_s^2 \: {[T^c]_f}^m \: {[T^c]_{f}}^n }{(2qq_j)} [\not{q}]\: 
[\not{q_i} ]
\end{split}
\end{equation}
finally, we achieve:
\begin{equation}
\begin{split}
|M_2|^2=\frac{g_s^2 \: {[T^c]_f}^m \: {[T^c]_{f}}^n }{(p_i \cdot p_j)}
[\not{p_i}][\not{p_j}]\otimes \frac{-(d-2)yz^2}{2(1-z)(1-y)}
\end{split}
\end{equation}
Interestingly, here is a term with $y$ concerning the gluon radiation from an anti-quark. This means that this result cannot contribute to the collinear limit for soft gluon $ y \rightarrow 0 $.
\newpage

\section{Interference contribution}
So far most of the work is done and we just have to put the results of $M_1$ and ${M_2}^{\dagger}$ next to each other, as we can see in the diagram. So we still get the interference contribution. 
\begin{figure}[h!]
\centering
\includegraphics[width=0.85\textwidth]{images/QQ/M1M2Degaqqg.png}
\end{figure}
Using the results from the previous section and we received:
\begin{equation}
\begin{split}
M_1\:{\color[RGB]{255,0,0} {M_2}^{\dagger}} = [{\bar{u}}_{\sigma}(q_i)\: (-ig_s \gamma^{\mu}\times {[T^a]_o}^l) \: \frac{i(\not{q_i} + \not{q})}{(q_i + q)^2}\:\: {\varepsilon^{\lambda_1}}_{\mu} (q)] [{v}_{\tau}(q_j)]\: \\
\quad\quad\quad\quad\quad\quad\quad\quad\:\:{\color[RGB]{255,0,0}[\bar{v}_{{\tau}^{\prime}}(q_j) \: (ig_s \gamma^{{\nu}^{\prime}}\times {[T^d]_{f^{\prime}}}^n) \: \frac{-i(\not{q_j} + \not{q})}{(q_j + q)^2} \: {\varepsilon^{\lambda_4}}_{{\nu}^{\prime}} (q)]\: [{u}_{{\sigma}^{\prime}}(q_i)]}
\end{split}
\end{equation}

%Kommentar

\begin{equation}
\begin{split}
M_1\: {M_2}^{\dagger} = \frac{g_s^2 {[T^a]_o}^l \:{[T^d]_{f^{\prime}}}^n }{(2q_i q)(2q_j q)} [\not{q_i}\: \gamma^{\mu} \: (\not{q_i} + \not{q})\: ]{\varepsilon^{\lambda_1}}_{\mu} (q) \: {\varepsilon^{\lambda_4}}_{{\nu}^{\prime}} (q)[\not{q_j} \:\gamma^{{\nu}^{\prime}} \: (\not{q_j} + \not{q})]\:
\end{split}
\end{equation}

\begin{equation}
\begin{split}
M_1\: {M_2}^{\dagger} = \frac{g_s^2 {[T^a]_o}^l \:{[T^a]_{f^{\prime}}}^n }{(2q_i q)(2q_j q)} [\not{q_i}\: \gamma^{\mu} \: (\not{q_i} + \not{q})\: ] (-g_{{\mu}{{\nu}^{\prime}}})[\not{q_j} \:\gamma^{{\nu}^{\prime}} \: (\not{q_j} + \not{q})]\:
\end{split}
\end{equation}



\begin{equation}
\begin{split}
M_1\: {M_2}^{\dagger} = \frac{-g_s^2 {[T^a]_o}^l \:{[T^a]_{f^{\prime}}}^n }{(2q_i q)(2q_j q)} [\not{q_i}\: \gamma^{\mu} \: (\not{q_i} + \not{q})\: ]
\:[\not{q_j} \:\gamma_{\mu} \: (\not{q_j} + \not{q})]\:
\end{split}
\end{equation}

%Kommentar 

Expectation:
\begin{figure}[h!]
\centering
\includegraphics[width=0.85\textwidth]{images/QQ/expectationM1M2dagger.png}
\end{figure}

\begin{equation}
\begin{split}
M_1\: {M_2}^{\dagger} =& \frac{-g_s^2 {[T^a]_o}^l \:{[T^a]_{f^{\prime}}}^n }{(2q_i q)(2q_j q)} [(z\not{p_i} + y(1-z)\not{p_j} + \sqrt{zy(1-z)}\not{m}_{\bot})\: \gamma^{\mu} \: (\not{p_i} + y\not{p_j)}]\\
&[(1-y) \not{p_j} \:\gamma_{\mu} \: ((1-z)\not{p_i} + (1+yz-y) \not{p_j} - \sqrt{zy(1-z)}\not{m}_{\bot})]\:
\end{split}
\end{equation}

\begin{equation}
\begin{split}
M_1\: {M_2}^{\dagger} =& \frac{-g_s^2 {[T^a]_o}^l \:{[T^a]_{f^{\prime}}}^n }{4(1-z)(1-y)y(1-2z+2z^2)(p_i \cdot p_j)(p_i \cdot p_j)} \\
&[z\not{p_i}\:\gamma^{\mu} \: \not{p_i} +zy\not{p_i}\:\gamma^{\mu}\:\not{p_j}+y(1-z)\not{p_j}\:\gamma^{\mu} \: \not{p_i}+y^2(1-z)\not{p_j}\:\gamma^{\mu} \:\not{p_j}\\
&+\sqrt{zy(1-z)}\not{m}_{\bot}\:\gamma^{\mu} \: \not{p_i}+y\sqrt{zy(1-z)}\not{m}_{\bot}\:\gamma^{\mu} \: \not{p_j}][(1-y)(1-z) \not{p_j} \:\gamma_{\mu} \:\not{p_i}\\&+(1-y)(1+yz-y) \not{p_j} \:\gamma_{\mu} \: \not{p_j}-(1-y)\sqrt{zy(1-z)} \not{p_j} \:\gamma_{\mu} \:\not{m}_{\bot})]
\end{split}
\end{equation}

\begin{equation}
\begin{split}
M_1\: {M_2}^{\dagger} =& \frac{-g_s^2 {[T^a]_o}^l \:{[T^a]_{f^{\prime}}}^n }{4(1-z)(1-y)y(1-2z+2z^2)(p_i \cdot p_j)(p_i \cdot p_j)} \\
&[z\not{p_i}\:\gamma^{\mu} \: \not{p_i} +zy\not{p_i}\:\gamma^{\mu}\:\not{p_j}][(1-z) \not{p_j} \:\gamma_{\mu} \:\not{p_i}+(1-y) \not{p_j} \:\gamma_{\mu} \: \not{p_j}]
\end{split}
\end{equation}

\begin{equation}
\begin{split}
M_1\: {M_2}^{\dagger} =& \frac{-g_s^2 {[T^a]_o}^l \:{[T^a]_{f^{\prime}}}^n }{4(1-z)(1-y)y(1-2z+2z^2)(p_i \cdot p_j)(p_i \cdot p_j)} [2z\:{p_i}^{\mu}\not{p_i} +2zy{p_j}^{\mu}\not{p_i}-zy\not{p_i}\not{p_j}\:\gamma^{\mu}]\\
&[2(1-z){p_i}_{\mu}\not{p_j}-(1-z)\not{p_j}\not{p_i}\:\gamma_{\mu}+2(1-y)\:{p_j}_{\mu}\not{p_j}]
\end{split}
\end{equation}

\begin{equation}
\begin{split}
M_1\: {M_2}^{\dagger} =& \frac{-g_s^2 {[T^a]_o}^l \:{[T^a]_{f^{\prime}}}^n }{4(1-z)(1-y)y(1-2z+2z^2)(p_i \cdot p_j)(p_i \cdot p_j)} \\
&[4z(1-z){p_i}^2\not{p_i}\not{p_j}-2z(1-z)\:\not{p_i}\not{p_j}\not{p_i}\not{p_i}+4z(1-y)(p_i \cdot p_j)\not{p_i}\not{p_j}+\\
&4zy(1-z)(p_j \cdot p_i)\not{p_i}\not{p_j}-2zy(1-z)\:\not{p_i}\not{p_j}\not{p_i}\not{p_j}+4zy(1-y){p_j}^2\not{p_i}\not{p_j}+\\
&-2zy(1-z)\not{p_i}\not{p_j}\not{p_i}\not{p_j}+zy(1-z)\not{p_i}\not{p_j}\gamma^{\mu}\not{p_j}\not{p_i}\:\gamma_{\mu}-2zy(1-y)\not{p_i}\not{p_j}\not{p_j}\not{p_i}]
\end{split}
\end{equation}

\begin{equation}
\begin{split}
M_1\: {M_2}^{\dagger} =& \frac{-g_s^2 {[T^a]_o}^l \:{[T^a]_{f^{\prime}}}^n }{y(1-2z+2z^2)(p_i \cdot p_j)} [\not{p_i}\not{p_j}]\otimes\frac{z}{1-z}
\end{split}
\end{equation}




Now we use the old parametrization to collect the singularities.



Here we can use the singular term in the denominator $ y(1-z) $ to drop the term with the same pre-factor and thus obtain:



\pagebreak

\section{Final result}
One could assume that for a complete result the contribution $ {M_1}^{\dagger} M_2 $ is still missing.
\begin{equation}
\lvert\:M\lvert^2\: = \lvert\:M_1\lvert^2\:+\lvert\:M_2\lvert^2\:+ M_1\: {M_2}^{\dagger} +{M_1}^{\dagger} M_2
\end{equation}
\begin{figure}[h!]
\centering
\includegraphics[width=0.85\textwidth]{images/QQ/qqgMSquer.png}
\end{figure}
It should be noted that it is completely sufficient to calculate $M_1\: {M_2}^{\dagger}$, because we know it from the quadratic amount of the complex numbers, we can calculate double of real part of $2RE(M_1\: {M_2}^{\dagger})$ instead of $ M_1\: {M_2}^{\dagger} +{M_1}^{\dagger} M_2 $ and that is exactly what is preferred here.
\begin{equation}
\lvert\:M\lvert^2\: = \lvert\:M_1\lvert^2\:+\lvert\:M_2\lvert^2\:+ {\color[RGB]{255,0,0} 2RE(M_1\: {M_2}^{\dagger})}
\end{equation}
\begin{figure}[h!]
\centering
\includegraphics[width=0.85\textwidth]{images/QQ/REqqgMSquer.png}
\end{figure}
Let's just add up the results from the previous sections and get:
\begin{equation}
\begin{split}
&\lvert\:M\lvert^2\: = (d-2)(1-z)(1-y)\:\frac{g_s^2  {[T^a]_{o}}^k \: {[T^a]_o}^l }{2y(1-2z+2z^2)(p_i \cdot p_j)}
[\not{p_i}][\not{p_j}]\\
&-(d-2)yz^2\:\frac{g_s^2 \: {[T^c]_f}^m \: {[T^c]_{f}}^n }{2(1-z)(1-y)(p_i \cdot p_j)}
[\not{p_i}][\not{p_j}]\\
&+2RE((\frac{-2z}{z-1}) \frac{g_s^2 \:\:{[T^a]_o}^l \:{[T^a]_{f}}^n }{2y(1-2z+2z^2)(p_i \cdot p_j)} 
[\not{p_i}][\not{p_j}])
\end{split}
\end{equation}
Now we use the knowledge from the introduction about the calculation of the Colour factor. With Fritz equation:
\begin{equation}
{T^a}_{o\:k} \: {T^a}_{l\:o} = \frac{1}{2}(\delta_{oo}\delta_{lk}-\frac{1}{N}\delta_{ok}\delta_{lo})= \frac{1}{2}(N\delta_{lk}-\frac{1}{N}\delta_{lk})=C_F \delta_{lk}
\end{equation}
After summation over the final colour states and averaging over initial colour states we get:

\begin{equation}
{T^a}_{o\:k} \: {T^a}_{l\:o}=C_F \delta_{lk}=\frac{1}{N} \displaystyle\sum\limits_{l=1}^ N \delta_{lk}C_F=C_F
\end{equation}
The same calculation for $ {T^c}_{m\:f} \: {T^c}_{f\:n} $ and $ {T^a}_{o\:l} \: {T^a}_{f\:n} $ turns $ C_F $ out as the colour factor.
Now we are going to compute the splitting function in the case of the colinearity, wich means, if:
\begin{equation}
y \longrightarrow 0
\end{equation}

%\begin{equation}
%\begin{split}
%&\lvert\:M\lvert^2\: = (d-2)(1-z)(1-y)\:\frac{g_s^2 C_F}{2y(1-2z+2z^2)(p_i \cdot p_j)}
%[\not{p_i}][\not{p_j}]\\
%&-(d-2)yz^2\:\frac{g_s^2 \: C_F }{2(1-z)(1-y)(p_i \cdot p_j)}
%[\not{p_i}][\not{p_j}]\\
%&+2RE((\frac{-2z}{z-1}) \frac{g_s^2 C_F}{2y(1-2z+2z^2)(p_i \cdot p_j)} 
%[\not{p_i}][\not{p_j}]
%\end{split}
%\end{equation}

\begin{equation}
\lvert\:M\lvert^2\: = \frac{g_s^2 C_F}{2y(1-2z+2z^2)(p_i \cdot p_j)}[\not{p_i}][\not{p_j}] \otimes((d-2)(1-z)-\frac{4z}{z-1})\:
\end{equation}
\\
for $ d=4-2\epsilon $

%\begin{equation}
%\begin{split}
%\lvert\:M\lvert^2\: = C_F((4-2\epsilon-2)(1-z)+\frac{4z}{1-z})\:\frac{g_s^2}{2y(1-2z+2z^2)(p_i \cdot p_j)}[\not{p_i}][\not{p_j}]\\
%=C_F(\frac{2(1-\epsilon)(1-z)^2+4z}{1-z})\:\frac{g_s^2}{2y(1-2z+2z^2)(p_i \cdot p_j)}[\not{p_i}][\not{p_j}]\\
%C_F(\frac{2-4z+2z^2-\epsilon(1-z)^2+4z}{1-z})\:\frac{g_s^2}{2y(1-2z+2z^2)(p_i \cdot p_j)}[\not{p_i}][\not{p_j}]\\
%=C_F(\frac{(1+z^2)}{1-z}-\epsilon(1-z))\:\frac{g_s^2}{y(1-2z+2z^2)(p_i \cdot p_j)}[\not{p_i}][\not{p_j}]\\
%=\langle\:\hat{P_{qq}}\rangle\:\frac{g_s^2}{q_i \cdot q}[\not{p_i}][\not{p_j}]\\
%\end{split}
%\end{equation}

\begin{equation}
\begin{split}
\lvert\:M\lvert^2\: &=\frac{g_s^2}{y(1-2z+2z^2)(p_i \cdot p_j)}[\not{p_i}][\not{p_j}]\otimes C_F(\frac{(1+z^2)}{1-z}-\epsilon(1-z))\\
&=\frac{g_s^2}{q_i \cdot q}[\not{p_i}][\not{p_j}]\otimes \langle\:\hat{P_{qq}}\rangle\:\\
\end{split}
\end{equation}

With Alterali-Parisi splitting function $ \langle\:\hat{P_{qq}}\rangle\: $ in the collinear limes, which was mentioned in the previous chapter. This is exactly the confirmation of our calculation that our calculation was actually performed correctly, otherwise we would not have received the same splitting function for soft gluons.
\newpage

\section{Double-check the results with the new kinematic}
One could do exactly the same calculation for the new kinematics to see if you get the same result in the collinear limit. From the next chapter we will explicitly work with the new parametrisation, because we found that the old kinematics only work in NLO and one-single emission. 
\subsection*{$ |M_1|^2 $}

\begin{equation}
\begin{split}
|M_1|^2=(d-2)\:\frac{g_s^2  C_F }{(2k_1\cdot q_i)}
[\not{k_1} ][\not{q_k}]
\end{split}
\end{equation}

\begin{equation}
\begin{split}
&|M_1|^2=(d-2)\:\frac{g_s^2 \: C_F }{2y\: p_i \cdot Q}
[(\alpha_1 -y\beta_1(\frac{Q^2}{2p_i \cdot Q})) \not{p_i} + y\beta_1\not{Q} + \sqrt{y\alpha_1\beta_1}\not{n}_{\bot,1} ]\\
&[A_1\not{p_i} + A_2\not{Q} + \sqrt{1-y}\not{p_k}]
\end{split}
\end{equation}

\begin{equation}
\begin{split}
&|M_1|^2=(d-2)\:\frac{g_s^2 \: C_F }{2y\: p_i \cdot Q}
[(A_2(\alpha_1 -y\beta_1(\frac{Q^2}{2p_i \cdot Q}))+ A_1y\beta_1) {p_i}\cdot Q\\
&+(\alpha_1 -y\beta_1(\frac{Q^2}{2p_i \cdot Q}))\sqrt{1-y}p_i\cdot p_k+A_2 y\beta_1 Q^2+ \sqrt{1-y}\sqrt{y\alpha_1\beta_1}{n}_{\bot,1}\cdot p_k ]\\
\end{split}
\end{equation}

For the collinearity $ y \rightarrow 0 $ we'll get:

\begin{equation}
\begin{split}
&|M_1|^2=(d-2)\:\frac{g_s^2 \: C_F }{2y\: p_i \cdot Q}
[(A_2(\alpha_1 -y\beta_1(\frac{Q^2}{2p_i \cdot Q}))+ A_1y\beta_1) \not{p_i} \not{Q}\\
&+(\alpha_1 -y\beta_1(\frac{Q^2}{2p_i \cdot Q}))\sqrt{1-y}\not{p_i} \not{p_k}+A_2 y\beta_1 Q^2+ \sqrt{1-y}\sqrt{y\alpha_1\beta_1}\not{n}_{\bot,1} \not{p_k} ]\\
\end{split}
\end{equation}

\begin{equation}
\begin{split}
&|M_1|^2=(d-2)(1-\beta_1)\sqrt{1-y}\:\frac{g_s^2 \: C_F }{2y\: p_i \cdot Q}
[\not{p_i} \not{p_k} ]\\
\end{split}
\end{equation}

\subsection*{$ |M_2|^2 $}

\begin{equation}
\begin{split}
|M_2|^2 =(d-2) \frac{g_s^2 \: C_F }{2k_1 \cdot q_k} [\not{k_1}]\: 
[\not{q_i} ]
\end{split}
\end{equation}

\begin{equation}
\begin{split}
&|M_2|^2 =(d-2) \frac{g_s^2 \: C_F}{2k_1 \cdot q_k} [(\alpha_1 -y\beta_1(\frac{Q^2}{2p_i \cdot Q})) \not{p_i} + y\beta_1\not{Q} + \sqrt{y\alpha_1\beta_1}\not{n}_{\bot,1}]\: \\
&[(\beta_1 -\alpha_1 y(\frac{Q^2}{2p_i \cdot Q}))\not{p_i} + y\alpha_1\not{Q} - \sqrt{y\alpha_1\beta_1}\not{n}_{\bot,l} ]
\end{split}
\end{equation}

Which means:
\begin{equation}
\begin{split}
&|M_2|^2 \sim(d-2) \frac{g_s^2 \: C_F}{2k_1 \cdot q_k} y[...]\\
&\:\:\:\:\:\:\:\:|M_2|^2\rightarrow 0 \:\:\:\:\:\:\:\text{for}\:\:\:\:\: y\rightarrow 0
\end{split}
\end{equation}

\subsection*{$ M_1\: {M_2}^{\dagger} $}


\begin{equation}
\begin{split}
&M_1\: {M_2}^{\dagger} = \frac{-g_s^2\: C_F }{4y(1-\beta_1) (1-y)\:(p_i \cdot p_k)(p_i \cdot Q)} \\
&4(\beta_1 \sqrt{1-y}{p_i}\cdot {{p_k}})[\beta_1 \sqrt{1-y}\not{p_i} \not{p_k}+ (1-\beta_1) \sqrt{1-y}\not{p_i}\not{p_k}]
\end{split}
\end{equation}

\begin{equation}
\begin{split}
&M_1\: {M_2}^{\dagger} = \frac{-g_s^2\: C_F }{y(1-\beta_1) \:(p_i \cdot p_k)(p_i \cdot Q)} \beta_1( {p_i}\cdot {{p_k}})[\beta_1 \not{p_i} \not{p_k}+ (1-\beta_1) \not{p_i}\not{p_k}]
\end{split}
\end{equation}

\begin{equation}
\begin{split}
&M_1\: {M_2}^{\dagger} = \frac{\beta_1}{(1-\beta_1)}\: \frac{-g_s^2\: C_F }{y \:(p_i \cdot Q)} [\not{p_i} \not{p_k}]
\end{split}
\end{equation}
\newpage