\section{Colour factor calculation}
fundamental representation in $ SU(2) $ and $ SU(3) $
\begin{equation}
\begin{split}
T^a = \tau^a \equiv \frac{\sigma ^2}{2}\:\:\:\:\:\:\: \mathit{with\: Pauli\: matrices\: \sigma ^a}\\
T^a = \vartheta^a \equiv \frac{\lambda ^2}{2} \:\:\:\:\:\:\: \mathit{with\: Gell-Mann\: matrices\: \lambda ^a}
\end{split}
\end{equation}

\begin{equation}
\begin{split}
\lambda ^1 =\begin{pmatrix} 0& 1 &\\ 1& 0 &\\ & & 0 \end{pmatrix},\:\:\: \lambda ^2 =\begin{pmatrix} 0& -i &\\ i& 0 &\\ & & 0 \end{pmatrix}, 
\:\:\: \lambda ^3 =\begin{pmatrix} 1&  &\\ & -1 &\\ & & 0 \end{pmatrix}, \:\:\: \lambda ^4 =\begin{pmatrix} &  &1\\ & 0&\\1 & &  \end{pmatrix}\\\
\lambda ^5 =\begin{pmatrix} &  &-i\\ & 0 &\\ i& &  \end{pmatrix},\:\:\: \lambda ^6 =\begin{pmatrix} 0&  &\\ & 0 &1\\ & 1& 0 \end{pmatrix}, 
\:\:\: \lambda ^7 =\begin{pmatrix} 0&  &\\ & 0 &-i\\ & i& 0 \end{pmatrix}, \:\:\: \lambda ^8 =\frac{1}{\sqrt3}\begin{pmatrix} 1&  &\\ & 1&\\ & &-2  \end{pmatrix}
\end{split}
\end{equation}
As we can see, $ {\lambda}^3 $ and $  {\lambda}^8 $ are diagonal.
These generators satisfy:
\begin{equation}
[T^a, T^b] = i \epsilon^{abc} T^c
\end{equation}

The most common convention for the normalization of the generators in physics is:
\begin{equation}
\displaystyle\sum\limits_{c,d} f^{acd} f^{bcd} = N \delta^{ab}
\end{equation}
The main relation we will use later for SU(N):
\begin{equation}
tr(T^a T^b)= {T_{ij}}^a {T_{ji}}^b = T_F \delta^{ab}
\end{equation}
\begin{equation}
\displaystyle\sum\limits_{a} (T^a T^a) = C_F \delta^{ij}
\end{equation}
\begin{equation}
f^{acd} f^{bcd} = C_A \delta^{ab}
\end{equation}
With $  T_F = \frac{1}{2} $ , $ C_A = N $ and $ C_F = \frac{N^2 -1}{2N} $.

\begin{equation}
f^{abc} = -2i tr(T^a[T^b, T^c])
\end{equation}
\begin{equation}
d^{abc} = 2 tr(T^a{T^b, T^c})
\end{equation}
\begin{equation}
T^a T^b = \frac{1}{2} (\frac{1}{N} \delta_{ab}+(d^{abc} + if^{abc})T^c)
\end{equation}
\begin{equation}
tr(T^a T^b T^c)= \frac{1}{4} (d^{abc}+if^{abc})
\end{equation}
\begin{equation}
tr(T^a T^b T^a T^c)= \frac{-1}{4N} \delta_{bc}
\end{equation}
\begin{equation}
f^{acd} f^{bcd}= N \delta^{ab}
\end{equation}
\begin{equation}
f^{acd} d^{bcd}= 0
\end{equation}
\begin{equation}
f^{ade} f^{bef} f^{cfd}= \frac{N}{2} f^{abc}
\end{equation}
Fierz identity:
\begin{equation}
\displaystyle\sum\limits_{a} {T_{ij}}^a {T_{kl}}^a = \frac{1}{2}(\delta_{il}\delta_{kj}-\frac{1}{N}\delta_{ij}\delta_{kl})
\end{equation}

%Casimir operators:
%\begin{equation}
%\begin{split}
%(T^a T^b)_{ij} = {T_{ik}}^a {T_{kj}}^a = \frac{1}{2}(\delta_{ij}\delta_{kk}-\frac{1}{N}\delta_{ik}\delta_{kj})\\
%\frac{1}{2}(\delta_{ij}N-\frac{1}{N}\delta_{ij})=\delta_{ij}\frac{N^2-1}{2N}=C_F \:\delta_{ij}\\ 
%\overset{in \:SU(3)}{=} \quad\begin{pmatrix} 4/3\:|r\bar{r}>& &\\ &4/3\:|g\bar{g}> & \\ & & 4/3\:|b\bar{b}>}\end{pmatrix}
%\end{split}
%\end{equation}
