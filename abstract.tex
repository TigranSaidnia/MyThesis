\thispagestyle{empty}
\section*{\Large \bfseries \centering Abstract}
\vspace{1cm}
Infra red divergences come in two flavours:soft, due to the  massless  nature  of  the  radiation  (e.g.  the  massless photon in QED), and collinear, which comes from treating the radiating particle as massless.
By the Catani-Seymour dipole factorization a $m+1$-parton matrix element can be written as product of $m$-parton matrix element and an universal singular splitting function.  
In this work a mapping for $ 3\rightarrow 2 $ will be outlined and following along those lines one for $ m+1\rightarrow m $ is proposed, which is explicitly evaluated for the quadratic matrix elements in terms of the four possible parton splitting in the soft and collinear regions. Furthermore, a general prescription for the simplification of the usage of this algorithm in the next-to-leading order (NLO) level will also be given.  For comparison, the known result from the $ e^{+}e^{-} \rightarrow q \bar{q} g $ process is compared with the result of the gluon radiation from a parten quark in the first part of chapter \ref{LO}. 
The algorithm is straightforwardly implementable in general purpose \textup{Mathematica} or \textup{Herwig++} \cite{Bahr:2008pv}.

\vspace{3cm}
\section*{\Large \bfseries \centering Zusammenfassung}
\vspace{1cm}
Infrarot-Divergenzen gibt es in zwei Varianten: soft, aufgrund der massenlosen Natur der Emission (z.B. das massenlose Photon in QED), und kollinear, das von der Behandlung des strahlenden Partikels als massenlos verursacht wird.
Durch die Catani-Seymour Dipolfaktorisierung kann ein $m+1$-Partonmatrixelement als Produkt aus einem $m$-Parton Matrixelement und einer universellen singul\"aren Splittingfunktion geschrieben werden.  
In dieser Arbeit wird eine Mapping für $ 3\rightarrow 2 $ und ebenso f\"ur eine $ m+1\rightarrow m $ pr\"asentiert, die dann explizit bei der Auswertung der Matrixelemente der vier möglichen Partonsplittingen in den soft und kollinearen Bereichen eingesetzt wird. Vielmehr wird eine allgemeine Prozedur für die Vereinfachung dieses Algorithmusses in der Next-to-Leading Order (NLO) Ebene vorgeschlagen.  Zum Vergleich wird das bekannte Ergebnis aus $ e^{+}e^{-} \rightarrow q \bar{q} g $ Prozess mit dem Ergebnis der Gluonenabstrahlung eines Parent-Quarks aus dem ersten Teil vom Kapitel \ref{LO} verglichen. 
Der Algorithmus kann in \textup{Mathematica} oder \textup{Herwig++} \cite{Bahr:2008pv} implementiert werden.