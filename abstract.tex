\section*{\Large \bfseries Abstract}
\vspace{3cm}
\section*{Identification of air-shower induced radio signals \\ in the AERA antenna stations}
For a deeper understanding of the origin of high-energy cosmic rays detailed information on the energy, the direction and the mass are necessary. There has been growing interest in the detection of high-energy cosmic rays via the radion emission produced in extensive air showers, i.e. the cascade of secondary particles induced by cosmic rays. An important component of the extensive air showers is the pair of electrons and protons which produce radio emission due to electrodynamic effects, like  by Geomagnetic- and Askaryan effects, which are subsequently discussed in detail.\\ 
Auger Engineering Radio Array (AERA) which is located at the Pierre Auger observatory in Argentina, aims to detect high-energy cosmic rays through the light of radio signals received by the AERA's antennas. The received signals are not free of noise.
Thus the separation of true signals from false-positive signals (caused by noise) is one of the major problems in the radio signal processing, thereby reconstructing the air showers.

The aim of this thesis is to investigate selective cuts on observable describing the true radio signals, in order to separate them from false-positive signals with high efficiency and purity.
\\
The main observables of the selection criteria are signal-to-noise ratio, signal time, and the angle between the measured and expected electric field vectors.
\\
By optimizing the cuts on the observables, we obtained radio signal selection  efficiency and purity of 88 \% and 99.6 \%, respectively.
