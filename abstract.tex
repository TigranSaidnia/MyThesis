\thispagestyle{empty}
\section*{\Large \bfseries \centering Abstract}
\vspace{1cm}

By the Catani-Seymour dipole factorization a $m+1$-parton matrix element can be written as product of $m$-parton matrix element and an universal singular splitting function.  
In this work a mapping for $ 3\rightarrow 2 $ will be outlined and following along those lines one for $ m+1\rightarrow m $ is proposed, which is explicitly evaluated for the evaluation of the matrix elements for the four possible parton splitting in the soft and collinear regions. A general prescription for the simplification of the usage of this algorithm in the next-to-leading order (NLO) level will also be given.  For comparison, the known result from the $ e^{+}e^{-} \rightarrow q \bar{q} g $ process is compared with the result from the gluon radiation from a parten quark. 
The algorithm is straightforwardly implementable in general purpose \textup{Mathematica} or Monte Carlo programs \textup{Herwig++} \cite{Bahr:2008pv}.

\vspace{3cm}
\section*{\Large \bfseries \centering Zusammenfassung}
\vspace{1cm}

Durch die Catani-Seymour Dipolfaktorisierung kann ein $m+1$-Partonmatrixelement als Produkt aus $m$-Partonmatrixelement und einer universellen singul\"aren Splittingfunktion geschrieben werden.  
In dieser Arbeit wird eine Mapping für $ 3\rightarrow 2 $ und genauso gut f\"ur $ m+1\rightarrow m $ pr\"asentiert, die dann explizit bei der Auswertung der Matrixelemente der vier möglichen Partonsplittingen in den soft und kollinearen Bereichen eingesetzt wird. Es wird auch ein allgemeine Prozedure für die Vereinfachung dieses Algorithmus in der Next-to-Leading Order (NLO) Ebene gegeben.  Zum Vergleich wird das bekannte Ergebnis des $ e^{+}e^{-} \rightarrow q \bar{q} g $ Prozesses mit dem Ergebnis der Gluonenstrahlung eines Parent-Quarks verglichen. 
Der Algorithmus kann in \textup{Mathematica} oder der Monte Carlo Simulation \textup{Herwig++} \cite{Bahr:2008pv} implementiert werden.