In this thesis a parton shower model based on Catani-Seymour dipole subtraction method was presented.
A $3\rightarrow 2 $-mapping and in this sense a $m+1\rightarrow m $-mapping for one single emission was proposed. With the parametrisation, the quadratic matrix elements were explicitly evaluated in terms of the four possible parton splittings in the soft and collinear areas. The kinematics could be used to verify the spin-averaged  unregularized  Altarelli-Parisi  splitting  functions  in $ d $-dimensions. In addition, a procedure was suggested to show how to concentrate on the important single terms for the calculation of the splitting function in order to waive the considerable and large terms. By this concept it was sufficient to simply sketch the final result, what was expected after the evaluation of the respective matrix elements. Thus, it was completely sufficient to ignore the other terms from the quadratic matrix element and concentrate on a single term. Finally, we took a close look at a well-known example, namely $ e^{+}e^{-} \rightarrow q \bar{q} g $, to compare the final result of Gluon emission from a parent (anti)quark \ref{fir} with the outcome \ref{confi} from this annihilation process. It was to be noted that in this work information about the behaviour of ${m^{\mu}}_{\bot}$ for the evaluation of the full matrix element was missing and moreover the full result from the annihilation process could not reproduce. Nevertheless, both results showed the same properties in the collinear region. For this goal a transformation was proposed so that the result could be shown in the picture of the respective parametrisations of the results from \ref{fin} and the $ e^{+}e^{-}$ annihilation. 

\section*{Future outlook}
Looking to the future, there are several avenues along which the present work could be continued. As a further procedure one must achieve the full result due to the kinematics and achieve numerically with a contour plot the same three partons configuration with the presented transformation. In this way it can be ensured that both results are in full agreement. For this aim there is a need about the product of ${m^{\mu}}_{\bot}$ with the other Momenta $ p_i, p_j $. This information can probably be obtained from the simulation program HERWIG++. Finally one can have a look at the double emissions.