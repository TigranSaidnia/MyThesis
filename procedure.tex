\section*{Concept}
\label{Concept}
\begin{itemize}
\item First of all, let's take a look at a possible spliting. For this we have to make sure that all possible meaningful diagrams have been considered.
All $ M_1 $, $ M_2 $, $ {M_1}^{\dagger} $ and $ {M_2}^{\dagger}$ diagrams are indexed independently of each other. When setting up the interference terms, however, we will cut out the respective results from the above-mentioned diagrams. To determine the matrix elements, we will use the Feynmann Rules from the appendix.
\item In the next step you have to decide for an observer with q. It is necessary to take care that the observer must be selected so that when the diagrams of M1 and M2 or M2 and M1 are next to each other, you get a full but cut diagram so that you can use the cutting rules. Whereby we don't use cutting rules here, but separate diagrams. This point will be clarified in the next chapter.
\item The square matrix element is a complex number and therefore:
\item we have to keep all indices as they are defined at $M_1$ and $M_2$. This is very important for the calculation of the interference term in order to get a meaningful result.
\item In the case of the gluon loop, you have to add the corresponding ghost diagram.
\item for all parts of the matrix element, use the parametrisation and the recipe for sending the kinematics from the last section.
After the addition of all terms we get all singularities (soft and collinear). 
\item To determine whether everything was calculated correctly, one goes to the collinear limits, so to speak for y, because here one has to get the Alterali-Parisi splitting function known in the parton shower
\end{itemize}