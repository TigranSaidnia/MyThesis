\section*{Concept}
\label{Concept}
Before the concept is explained, it should be mentioned at this point that the steps are gradually explained in more detail in the next steps. This provides a rough overview and can be used as a reference for the other sections.
\renewcommand{\labelenumi}{\roman{enumi})}
\begin{enumerate}
\item First of all, let's take a look at a possible spliting. For this we have to make sure that all possible meaningful diagrams have been considered.
All $ M_1 $, $ M_2 $, $ {M_1}^{\dagger} M_2 $ and $ M_1{M_2}^{\dagger}$ diagrams are indexed independently of each other.  To determine the matrix elements, we will use the Feynmann Rules which is explained in detail in section \ref{QCD Lagrangian}.
\item Each diagram consists of an emitter and a spactator part.
The emitter part itself contains an emitter parent with the impulse $ q_i+q $ , of which a patron is split with $ q $ . A daughter-patron with $ q_i $ remains of the parent patron. One should select the spectator skilfully, so that the diagrams are meaningful and calculable in the case of the interfrence terms, otherwise one must manipulate with the final results because of the unanimity of the indices. Thus, one has a structure and one can place the diagrams from $ M_1, {M_1}^{\dagger}, {M_2}^{\dagger}$ and $ M_2 $ side by side and even use their probability amplitude for the interference terms without having to recalculate them every time. 
\item The square matrix element is a complex number and therefore:
\item we have to keep all indices as they are defined at $M_1$ and $M_2$. This is very important for the calculation of the interference term in order to get a meaningful result.
\item In the case of the gluon loop, you have to add the corresponding ghost diagram.
\item for all parts of the matrix element, use the parametrisation and the recipe for sending the kinematics from the last section.
After the addition of all terms we get all singularities (soft and collinear). 
\item To determine whether everything was calculated correctly, one goes to the collinear limits, so to speak for y, because here one has to get the Alterali-Parisi splitting function known in the parton shower
\end{enumerate}