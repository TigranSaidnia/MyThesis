\section*{Concept}
\label{Concept}
Before the procedure is explained, it should be mentioned at this point that the steps are gradually explained in more detail in the next steps. This only provides a rough overview and can be used as a reference for the other sections.
\renewcommand{\labelenumi}{\roman{enumi})}
\begin{enumerate}
\item First of all, look at a possible spliting. For this one has to make sure that all possible meaningful diagrams have been considered.
All $ M_1 $, $ M_2 $, $ {M_1}^{\dagger} M_2 $ and $ M_1{M_2}^{\dagger}$ diagrams are indexed independently of each other.  To determine the matrix elements, the Feynmann Rules will be used which is explained in detail in section \ref{QCD Lagrangian}.
\item Each diagram consists of an emitter and a spactator part.
The emitter part itself contains an emitter parent with the momentum $ q_i+q $ , of which a patron is split with $ q $. A daughter-patron with $ q_i $ remains of the parent patron. One should select the spectator skilfully, so that the diagrams are meaningful and calculable in the case of the interference terms, otherwise one must manipulate with the final results because of the unanimity of the indices. Thus, one has a structure and one can place the diagrams from $ M_1, {M_1}^{\dagger}, {M_2}^{\dagger}$ and $ M_2 $ side by side and even use their probability amplitude for the interference terms without having to recalculate them every time. 
\item Before starting to calculate, it will be tried to predict the expected result based on the contracting indices. Usually the non-contracting indices that remain form the final result as one or more tensors. This is relatively helpful when a calculation for a certain limit is performed, because it can be quickly seen from the square matrix element which terms need to be calculated.
This will become clearer later if you want to determine the splitting functions in the collinear limit from the respective diagram.
\item finally, to determine whether everything was calculated correctly, one goes to the collinear limits because it is known that in this case the well known Alterali-Parisi \ref{Alterali-Parisi} splitting function have to be output.
\end{enumerate}