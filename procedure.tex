\section*{Concept}
\label{Concept}

Before the procedure is explained it should be mentioned the steps are gradually explained in more detail in the next chapter. This only provides a rough overview and can be used as a reference for the computing.
\\
\renewcommand{\labelenumi}{\roman{enumi})}
\begin{enumerate}
\item Considering a possible splitting. For this one has to ensure that each possible meaningful diagrams have been considered.
All $ M_1 $, $ M_2 $, $ {M_1}^{\dagger} M_2 $ and $ M_1{M_2}^{\dagger}$ diagrams need to be indexed independently.  To determine the matrix elements, Feynmann rules will be used which is derived in detail in the appendix section \ref{Feynman}. Before the kinematics is used, the obtained matrix element should be simplified by matrix algebra, otherwise the calculation becomes clearly more complicated.
\\
\item Each diagram consists of two emitters and a spectator.
The emitter parent with the momentum $ q_i+q $ in the old kinematic splits into a daughter-patron with $ q_i $ and an another emitter $ q $. One should select the spectator $ q_j $ skilfully so that the diagrams are meaningful and calculable in the case of the interference terms, otherwise the final results hast to be changed cause of the unanimity  indices. Thus a structure is achieved and the diagrams can be replaced from $ M_1, {M_1}^{\dagger}$,  $ M_2 $ and ${M_2}^{\dagger}$ step by step and even their amplitudes can be used for the interference terms. 
\\
\item Before starting to calculate, based on the contracted indices the expected result should be predicted. This is relatively helpful since it can be quickly seen from the square matrix element which terms are relevant for the splitting functions.
\\
\item It is recommended to use the concepts from the previous section \ref{recepie}. Basically, there are four common scalar products from the denominators listed here:

\begin{itemize}
\item $ (q \cdot q_i)(q \cdot q_i) $
\item $ (q \cdot q_j)(k_1 \cdot q_j) $
\item $ (q \cdot q_i)(q \cdot q_j) $
\item $ (q \cdot q_i)(q_i \cdot q_i) $
\end{itemize}

For the new parametrisation, the following substitution is used:
\begin{equation}
\begin{split}
&q_i \rightarrow q_i\\
&q \: \rightarrow k_1\\
&q_j \rightarrow q_k
\end{split}
\end{equation}
\\
\item finally, to find out whether everything was calculated correctly, the collinear limits checks if the Alterali-Parisi splitting functions  \ref{Alterali-Parisi} are output.
\\
\item In the case of indistinguishable particles in relation to the interference term, the momentum of the particles for the same diagram must be exchanged once.
\end{enumerate}
