
Knowledge is a human need. For thousands of years we have been trying to understand the secrets of the universe. Such riddles fascinated even Johann Wolfgang von Goethe, as he wrote in his book Faust \cite{goethe1921faust}; eine Tragedie, "What holds the world together in its innermost." 
Almost 400 years before Christ, an ancient Greek philosopher, Democritus, and his teacher Leukipp claimed that matter cannot be divided at will. Rather, there must be an Atomos (Greek: indivisible) that could no longer be subdivided.
Democritus was of the opinion that there were infinitely many atoms with different geometric forms that were in contact in a certain way. He pointed out that a thing has a color, taste or even soul, based on the apparent effect of the composition of these small grains.
\cite{capelle1968vorsokratiker}

This statement of Democritus was first laughed at by the renowned philosopher Aristotiles. It took about 2000 years for a chemist named John Dalton to deal with the subject. Based on various test series, he summarized his conclusion in his book A New System of Chemical Philosophy, that all substances consist of spherical indivisible atoms. The atoms of different elements have different masses and volumes. This was exactly the most striking difference to Democritus's atomic world.\cite{dalton2010new}

The discovery of the periodic system by D. Mendeleev and P. Meyer enabled us to arrange the atoms according to their mass in such a way that their properties occur in a certain order.\cite{haken2013atom}

In 1897 Joseph Thompson was able to obtain a stream of particles by heating metals and deflecting them by a magnetic field. This electron beam was 200 times lighter than the lightest atom, hydrogen.
His conclusion was that atoms cannot be indivisible. He suggested that each atom consists of an electrically positively charged sphere in which electrically negatively charged electrons are stored - like raisins in a cake.

furthermore, renowned scientists as well as Marie and Pierre Curie have contributed much to the development of atomic theory by discovering radioactivity, Boltzmann by kinetic gas theory and Plank, the founder of quantum physics.
However, one of the most important steps in the atomic model was taken by the British physicist Rutherford. He bombarded a thin aluminium foil with a radioactive sample. If Thompson's cake model were correct, only a few alpha particles would be detected behind the aluminium foil. Surprisingly, many particles were visible, which could only be explained by the assumption that the majority of atoms consisted of empty spaces. Another miracle was that some particles could be seen above or below the target sample. Since it is known, the alpha particles are positively charged, it could be assumed the electric repulsive force of two positive charges. From the ideas of Planck and Rutherford, the Danish physicist Bohr (1885-1962) developed a planetary atomic model. The electrons then move around the nucleus in certain orbits, like planets orbit the sun. The orbits are also called shells. The special thing about it was that the distances of the electron orbits follow strict mathematical laws.\\  
At first, however, it remained unclear what this core should consist of. \cite{haken2013atom, demtroder2005experimentalphysik}   
In 1912, the Austrian physicist Victor Hess discovered during his balloon flights that the ionization rate of the Earth's atmosphere increases with altitude. This result was not expected because until then the Earth's radioactivity was known as the only source of air ionization. Therefore, he postulated this new type of radiation as cosmic radiation, which must originate outside the Earth's atmosphere ~\cite{Ender}.\\
Further investigations two years later confirmed the thesis of a cosmic background of such radiation. After this new discovery, it was discovered that the radiation consists of charged particles. In 1932, the American physicist Carl David Anderson was able to prove the postulated particle of Dirac, the positron, as a component of an air shower through his cloud chamber. For a long time, cosmic rays were the only way to analyse such exotic particles.\cite{Bluemer:2009zf}
This changed when particle accelerators were able to generate particles in collisions. But even today, cosmic rays are the only way to study particles of the highest energies, since these energies cannot be reached by today's particle accelerators, such as the LHC. The LHC, the world's largest accelerator at CERN, produces particles with centre-of-mass energy equivalent to a cosmic particle of nearly $10^{17} eV $, with the energy spectrum of cosmic particles reaching up to $10^{20} eV $.
However, we can only analyse such exotic particles in detail by increasing the luminosity and procession of the particle accelerators at the nucleus. 
The discovery of the neutron by Chadwick (1932) showed that atomic nuclei are made up of protons and neutrons. It was also clear that, in addition to gravitation and the electromagnetic force,there should exist two short-range forces in nature; a strong force which binds the nucleons together and a weak force which is responsible for radioactive.
In the meantime it was agreed that a new theory was needed for the classification and grouping of this particle zoo. This is how the current standard model came into being.\\
The SLAC experiments indicate that the electrons can be scattered as quasi-free point-like constituents within the proton structure, which actually meant that the protons or neutrons are not point-like and must consist of other constituents. Through the bubble chamber a huge number of previously invisible particles (Gell-Mann's eightfold path) could suddenly be made visible, which represented contradictions to the previous physics. To explain this, the physicist Gell-Mann found basic building blocks from which all previously known atomic particles should be built. The components are later identified with quarks.\\
For the results of particle physics experiments in connection with the strong interaction the perturbation theory is used. In order to make useful and more accurate predictions, the calculations must be carried out at least in next-to-leading order, in order to avoid large uncertainties arising from the non-physical scale dependencies.   
In this work, the dipole subtraction method will be used for calculating next-to-leading order splitting functions of parton shower in QCD.\\
Chapter \ref{Introduction} provides a brief overview of of the general method, describing the subtraction procedure and presenting the dipole formulae. Thereafter, the kinematics for the case of massless  partons with the useful prescriptions for the matrix elements evaluation are discussed. The factorization properties of QCD matrix elements in the soft and collinear limits for four possible parton shower due to parametrisations is outlined in \ref{LO}. 
Chapter \ref{Summary} represents  a  summary  of  the  previous final results  that  were  obtained. MATHEMATICAL TOOLS \ref{Math} gives more details, and some examples, of the necessary mathematically formulae for the handling of parametrisations. All detailed steps of the calculations can be found in the appendix \ref{Appendix}.

 