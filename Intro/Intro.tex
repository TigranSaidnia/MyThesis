\section{Brief history of particle physics}
Knowledge is a human need. For thousands of years we have been trying to understand the secrets of the universe. Such riddles fascinated even Johann Wolfgang von Goethe, as he wrote in his book Faust chapter 4 ; eine Tragedie, "What holds the world together in its innermost." 
Almost 400 years before Christ, an ancient Greek philosopher, Democritus, and his teacher Leukipp claimed that matter cannot be divided at will. Rather, there must be an Atomos (Greek: indivisible) that could no longer be subdivided.
Democritus was of the opinion that there were infinitely many atoms with different geometric forms that were in contact in a certain way. He pointed out that a thing has a color, taste or even soul, based on the apparent effect of the composition of these small grains.
Wilhelm Capelle: Die Vorsokratiker, Leipzig 1935, S. 399.

This statement of Democritus was first laughed at by the renowned philosopher aristotiles. It took about 2000 years for a chemist named John Dalton to deal with the subject. Based on various test series, he summarized his conclusion in his book A New System of Chemical Philosophy, that all substances consist of spherical indivisible atoms. The atoms of different elements have different masses and volumes. This was exactly the most striking difference to Democritus's atomic world.
A New System of Chemical Philosophy, Band 1, Teil 1, Manchester, London 1808,

The discovery of the periodic system by D. Mendeleev and P. Meyer enabled us to arrange the atoms according to their mass in such a way that their properties occur in a certain order.

In 1897 Joseph Thompson was able to obtain a stream of particles by heating metals and deflecting them by a magnetic field. This electron beam was 200 times lighter than the lightest atom, hydrogen.
His conclusion was that atoms cannot be indivisible. He suggested that each atom consists of an electrically positively charged sphere in which electrically negatively charged electrons are stored - like raisins in a cake.

furthermore, renowned scientists as well as Marie and Pierre Curie have contributed much to the development of atomic theory by discovering radioactivity, Boltzmann by kinetic gas theory and M. Plank, the founder of quantum physics.
However, one of the most important steps in the atomic model was taken by the British physicist E. Rutherford. He bombarded a thin aluminium foil with a radioactive sample. If Thompson's cake model were correct, only a few alpha particles would be detected behind the aluminium foil. Surprisingly, many particles were visible, which could only be explained by the assumption that the majority of atoms consisted of empty spaces. Another miracle was that some particles could be seen above or below the target sample. Since we knew that the alpha particles were positively charged, we could assume the electric repulsive force of two positive charges. In 1911, RUTHERFORD created the planetary model of the atom, which was developed a year later by his pupil NIELS BOHR (1885-1962) into a model known as the Bohr atom model.
At first, however, it remained unclear what this core should consist of.    
In 1912, the Austrian physicist Victor Hess discovered during his balloon flights that the ionization rate of the Earth's atmosphere increases with altitude. This result was not expected because until then the Earth's radioactivity was known as the only source of air ionization. Therefore, he postulated this new type of radiation as cosmic radiation, which must originate outside the Earth's atmosphere ~\cite{Ender}.\\
Further investigations two years later confirmed the thesis of a cosmic background of such radiation. After this new discovery, it was discovered that the radiation consists of charged particles. In 1932, the American physicist Carl David Anderson was able to prove the postulated particle of Dirac, the positron, as a component of an air shower through his cloud chamber. For a long time, cosmic rays were the only way to analyze such exotic particles.
This changed when particle accelerators were able to generate particles in collisions. But even today, cosmic rays are the only way to study particles of the highest energies, since these energies cannot be reached by today's particle accelerators, such as the LHC. The LHC, the world's largest accelerator at CERN, produces particles with centre-of-mass energy equivalent to a cosmic particle of nearly $10^{17} eV $, with the energy spectrum of cosmic particles reaching up to $10^{20} eV $.
However, we can only analyze such exotic particles in detail by increasing the luminosity and procession of the particle accelerators at the nucleus. 
The discovery of the neutron by Chadwick (1932) showed thatatomic nuclei are made up of protons and neutrons. It was alsoclear that, in addition to gravitation and the electromagnetic force,there should exist two short-range forces in nature: a strong forcewhich binds the nucleons together and a weak force which is respon-sible for radioactive.
In the meantime it was agreed that a new theory was needed for the classification and grouping of this particle zoo. This is how the current standard model came into being.

\section{Standard model}

\section{Quantum chromo dynamics}

Nowadays, we know there are four types of interactions, see below:\\


\begin{tabular}{|c|c|c|c|c|}
\hline 
Interaction & Energy scale & Range & Mediators \\ 
\hline 
Strong & $ \sim 1 $  & $10^{-15}=1 Fermi$ & $g$ \\ 
\hline 
Electromagnetic & $ \sim 10^{-2} $ & $\infty$ & $\gamma $ \\ 
\hline  
Weak & $ \sim 10^{-6} $ & $10^{-18}$ & $W^{\pm}, Z$ \\ 
\hline
Gravity & $ \sim 10^{-38} $ & $\infty$ & maybe graviton \\ 
\hline 
\end{tabular}  
\\
\\
\\
Otherwise, it's clear meanwhile that nucleons are made up of quark and gluons.
Whereby, the gluons are the exchange bosons for this short interaction.
\begin{figure}[h!]
\centering
\includegraphics[scale=0.7]{images/Neutron.png}
\caption{That's a schematic picture of neutron structure. at the left side of the diagram is the low-resolution to see. The 3 quarks picture allows us to interpretate the quantum numbers of the neutron in the valence band.
We also obtain a high-resolution picture for a large $ Q^2 $. Here we have a lot of gluons (gluon sea) and quarks pair.\\ 
The interesting thing is, it doesn't matter in which energy scale we observe the quantum number of a neutron, because it is always the same.}
\end{figure}
Quarks have not yet been observed as free particles. With increasing separation it will be easier to produce quark-antiquark pair than to isolate quark because the coupling between them too strong is.
Quarks prefer to bind into hadrons what can be classified to baryons with three quarks state and mesons with a quark-antiquark state.
As we know, the wave function of fermions must be antisymmetric according to Pauli exclusion principle under the exchange of two quarks. Interestingly, there are resonance states with spin $ \frac{3}{2} $ like $ {\Delta}^{++} $.
The spins of the three up quarks are parallel to each other, have the same flavour and orbital angular momentum L=0. This means that an exchange of flavour, spin and space (orbital angular momentum) does not lead to any change. This problem is solved with the additional degree of freedom, the so-called color charge. If the wave function is odd in color, we have solved user spin statistical problem.
The total wave function for each particle can be expressed in terms of:

\begin{equation}
\begin{split}
\Psi_{3q} &= \psi_{space} \times \chi_{spin} \times \theta_{colour} \times \phi_{flavour} \\
&\:\:\:\:\:\:\:O(3) \:\:\:\:\: SU(2)\:\:\:\: SU(3)\:\:\:\:\: SU(6)\\
\end{split}
\end{equation}
Now we can compute all possible States in regard to colour With Young Tableaux.
\begin{figure}[h!]
\centering
\includegraphics[scale=0.7]{images/Young.png}
\end{figure}



The quantum field theory which describes this area is called Quantum chromo dynamics short QCD.
QCD like QED and the weak interaction theory is described by representations of a symmetry group. From the condition that the Lagrangian must be invariant under arbitrary global and local symmetry transformations follows the interactions terms.

\newpage