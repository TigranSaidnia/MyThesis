
\section*{\Large \bfseries Introduction}

\vspace{3cm}


\section*{Emission kernel of parton shower}
For the results of particle physics experiments in connection with the strong interaction the perturbation theory is used. In order to make useful and more accurate predictions, the calculations must be carried out at least in next-to-leading order, in order to avoid large uncertainties arising from the unphysical scale dependencies. The combination of Parton shower simulations with corrections obtained by matrix elements of leading order and the addition of NLO to the production process has been investigated in the past.  
In this work, the dipole subtraction method will be used for calculating next-to-leading order corrections in QCD.



Durch die Catani-Seymour dipole factorization an m+1 parton matrix element can be written as product an m-parton matrix element and an universal singular factor.  
In this paper we outline eine Mapping für 3 -> 2 und genauso gut für m+1 -> m vorgeschlagen, die explizit für die Auswertung der Matrixelemente für die vier mögliche Partonsplliting in the soft- sowie collineare Regions ausgewertet wird. It wiil also given a general prescription for the simplification of the usage this algorithm in the next-to-leading order (NLO) level.  
The algorithm is straightforwardly implementable in general purpose Monte Carlo programs.




https://arxiv.org/pdf/hep-ph/0601021.pdf


 We present a new general algorithm for calculating arbitrary jet cross sections in arbitrary scattering processes to next-to-leading accuracy in perturbative QCD. The algorithm is based on the subtraction method. The key ingredi-ents are new factorization formulae, called dipole formulae, which implementin a Lorentz covariant way both the usual soft and collinear approximations,smoothly interpolating the two. The corresponding dipole phase space obeysexact factorization, so that the dipole contributions to the cross section can beexactly integrated analytically over the whole of phase space. We obtain explicitanalytic results for any jet observable in any scattering orfragmentation processin lepton, lepton-hadron or hadron-hadron collisions. Allthe analytical formu-lae necessary to construct a numerical program for next-to-leading order QCDcalculations are provided. The algorithm is straightforwardly implementable ingeneral purpose Monte Carlo programs.



Unfortunately, standard NLO programs have significant flaws. One flawis that the final states consist just of a few partons, while in nature finalstates consist of many hadrons. A worse flaw is that the weights are oftenvery large positive numbers or very large negative numbers.There is another class of calculational tools, the shower Monte Carloevent generators, such as HERWIG[1] and PYTHIA[2]. These have thesignificant advantage that the objects in the final state consist of hadrons


We begin in Sect. 2 by giving a brief overview of the general method, describing thesubtraction procedure and how our dipole formulae are used to implement it. In Sect. 3 weestablish the notation used throughout the paper. In Sect. 4we review the factorizationproperties of QCD matrix elements in the soft and collinear limits before presenting, inSect. 5, our dipole factorization formulae, which smoothlyinterpolate these two limitingregions. After briefly recalling, in Sect. 6, the precise definitions of QCD cross sectionsat NLO, we go on to describe in detail our subtraction method for evaluating these crosssections, in Sects. 7–11. In Sect. 12 we summarize and discuss our results. Appendix Agives more details, and some examples, of the necessary colour algebra. In Appendix B weexplicitly perform the only difficult integral we encounter.In Appendix C we collect to-gether the main formulae needed to implement our method in specific calculations. Finallyin Appendix D we work through a few simple examples of applying our method to specificcross sections